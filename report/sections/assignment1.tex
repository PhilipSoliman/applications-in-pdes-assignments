\section{Assignment 1}

% table of variables and their meanings
\begin{table}[H]
    \centering
    \begin{tabular}{|c|c|c|c|}
        \hline
        Variable & Unit &  Meaning & Value\\
        \hline
        $\theta$ & - & latitude & $[-\frac{\pi}{2}, \frac{\pi}{2}]$\\
        $t $ & $s$ & time & -\\
        $x = \sin \theta$ & -  & latitude coordinate & $[-1, 1]$\\
        $T $ & $K$ & temperature & -\\
        $R_A $ & $J s^{-1} m^{-2}$ & effective solar radiation & - \\
        $Q $ & $J s^{-1} m^{-2}$ & solar radiation & -\\
        $Q_0 $ & $J s^{-1} m^{-2}$ & solar radiation constant & 341.3\\
        $\alpha$ & - & albedo of Earth & -\\
        $\alpha_1$ & - & albedo of ice & 0.7\\
        $\alpha_2$ & - & albedo of water & 0.289\\
        $T^{*} K$ & - & temperature at which ice melts & 273.15\\
        $M$ & $K^{-1}$ & temperature gradient (?) & -\\
        $\mu $ & $J s^{-1} m^{-2}$ & greenhouse gas \& fine particle parameter & 30\\
        $R_E $ & $J s^{-1} m^{-2}$ & black body radiation & -\\
        $\epsilon_0$ & - & emmisivity of Earth & 0.61\\
        $\sigma_0 $ & $J s^{-1} m^{-2} K^{-4}$ & Stefan-Boltzmann constant & $5.67 \cdot 10^{-8}$\\
        $R_D $ & $J s^{-1} m^{-2}$ & heat dispersion & -\\
        $D $ & $J s^{-1} m^{-2}$ & heat dispersion constant & 0.3\\
        $\delta $ & $J s^{-1} m^{-2}$ & heat dispersion at poles & 0\\
        $C_T $ & $J K^{-1}$  & heat capacity of Earth & $5 \cdot 10^{8}$\\
        \hline
    \end{tabular}
    \caption{Variables and their meanings}
    \label{tab:vars}
\end{table}

\subsection{Introduction}
    $\delta$ cannot be positive (resp. negative) at $x = \pm 1$ (poles), otherwise energy would be
    artificially entering (resp. leaving) the system. Simply said, the poles cannot be a source or sink of energy.
    This requires us to set $\delta = 0$ at $x = \pm 1$. Furthemore 
    \begin{align*}
            \left.\frac{d T}{dx}\right|_{x = \pm 1} = 0.
    \label{eq:bcs}
    \end{align*}
    However, we run into a problem when we combine the boundary conditions and set $\delta = 0$. The 
    equation for the heat dispersion vanishes at the boundary and we are left with a zeroth order differential equation 
    for which we simply cannot satisfy, one let alone two, boundary conditions.

    Hence we must resort to a more basic requirement. Namely, that there exists an equilibrium temperature $T_0$ at the poles,
    \begin{align*}
        &\left.F(T(x, t))\right|_{x=\pm 1} = 0 \forall t > 0,\\
        &\implies \left.R_D\right|_{x=\pm 1} = -(R_A - R_E).
    \end{align*}
    Assuming we use the given expansion of T in terms of legendre polynomials, we can write the above Assignment
    \begin{align*}
        \left.\frac{d T}{dt}\right|_{x=\pm 1} &= \sum_{n=0}^{\infty} \left.\frac{d \phi_n}{dt}\right|_{x=\pm 1} P_n(x) = -(R_A - R_E).
    \end{align*}


\subsection{Embedding}
    See the end of section 4.4 in the book on how to construct possible embeddings.

\subsection{Conclusion}